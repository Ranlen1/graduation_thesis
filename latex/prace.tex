%ablona pro maturitní práci Gymnázia Jírovcova 8, České Budějovice
% Autoři šablony: Jonáš Havelka, Michal Kočer, Daniel Sýkora
% Typ dokumentu: report
% veškeré úpravy v soubor MP.sty (styl maturitní práce)
\documentclass[12pt]{report}
% %%%%%%%%%%%%%%%%%%%%%%%%%%%%%%%%%%%%%%%%%%%%%%%%%%%%%%%
\usepackage{config}						  % Import stylu maturitní práce
\author{Arnošt Thor}                  % AUTOR PRÁCE
\title{Jak psát práci v \LaTeX{}u}    % NÁZEV PRÁCE
\date{14. února 2025}                 % DATUM ODEVZDÁNÍ PRÁCE
\vedouci{Mgr. František Kantor, PhD.} % VEDOUCÍ PRÁCE
\place{V Českých Budějovicích}
\skolnirok{2025/2026}                  % ŠKOLNí ROK
\logo{\includegraphics[scale=1.25]{GJ8_logotyp}} %Logo školy
%%%%%%%%%%%%%%%%%%%%%%%%%%%%%%%%%%%%%%%%%%%%%%%%%%%%%%%%%%%%%%%%%%%
\begin{document} %%%%%%% začátek dokumentu
%%%%%%%%%%%%%%%%%%%%%%%%%%%%  Titulní stránka + úvodní povinné stránky
\pagenumbering{roman}                   % číslování stránek římskými číslicemi
	\mytitlepage						% Vygenerování titulní strany
	
	\prohlaseni{
		Prohlašuji, že jsem tuto práci vypracoval samostatně s vyznačením všech použitých pramenů.
	}	
	
	\abstrakt{
	}{
	}
	
	\podekovani{
	}
	
   {\tableofcontents\newpage}			% Obsah
	
\addtocounter{page}{1}		% Posunutí countru stránek
\pagenumbering{arabic}		% Číslování stránek arabskými číslicemi
\chapter*{Úvod}     % úvod práce 
	
%%%%%%%%%%%%%% TEORETICKÁ ČÁST %%%%%%%%%%%%%%%%%%	
\part{Teorie k vývoji hry v OpenGL}  % název teoretické části (nenechávejte Teoretická část)
	
\chapter{Programovací jazyky}
			
\section{Programovací jazyk C}
    C je středněúrovňový programovací jazyk, tedy jazyk, který je podobou blízko strojovému kódu,
    ale má už prvky vyššího programovacího jazyka jako jsou funkce, datové struktury nebo to že je
    strukturovaný. Je kompilovaný a statický, což znamená, že se program musí nejdříve přeložit do
    strojového kódu a až pak se může spustit. Datové typy jsou známy v čase kompilace, proto všechny
    proměnné musí být v kódu deklarovány, jelikož vkládání vstupních dat do programu probíhá až při
    běhu programu. C nemá automatický správce paměti, takže je potřeba uvolňovat paměť manuálně. C
    má střídmou standardní knihovnu, která obsahuje základní matematické operace a funkce pro
    práci s pamětí a soubory, takže jakékoliv složitější datové struktury či funkce si člověk musí
    naprogramovat sám. Tato strohost a blízkost ke strojovému kódu z C dělá jeden z nejrychlejších
    programovacích jazyků.
    \cite{programming_in_c, the_c_programming_language, features_of_C,
    what_is_difference_between_a_low-level_midlvel_and_high-level_language}\\ \\
    C bylo vytvořeno Dennisem Ritchiem na počátku 70. let 20. století v AT\&T Bellových 
    laboratořích. Jeho předchůdci byly jazyky ALGOL, CPL, BCPL a B. Jeho prvotním účelem bylo
    přepsat operační systém UNIX do použitelnějšího jazyka než Assembly a B. Už koncem 70. let bylo
    C populární, ale nebylo standardizované a vznikalo mnoho různých variant. Na začátku 80. let
    tedy Americký národní institut pro standardy(ANSI) zahájil práci na formální standardizované
    verzi. Tu dokončili v roce 1989 a je známa pod jménem C89. V průběhu let vycházely další verze,
    které jazyk zlepšovaly a modernizovali. Nejdůležitější verze byly C99, C11 a C17. Norma C23 byla
    nedávno schválena a teď se implementuje do kompilátorů. V součastnosti mezi nejpoužívanější
    kompilátory patří GCC a Clang. Jelikož bylo C velice populární,
    ovlivnilo řadu jiných programovacích jazyků, jako C++, C#, Java, Rust, Go {\gls{atd}}
    \cite{the_c_book:_featuring_the_ansi_c_standart, c:_a_reference_manual,
    programming_in_c, the_c_programming_language}\\ \\
    C je univerzální programovací jazyk, má tedy širokou škálu využití. Jeho první využití bylo k
    napsání UNIXu, který později ovlivnil operační systémy jako Linux, macOS, iOS a Android.
    Používá se v programování softwaru s omezenou pamětí a výkonem, jako je firmware aut či v
    zařízeních chytrých domácností. Dále se využívá pro tvoření kompilátorů a interpreterů jako je
    GCC nebo interpreter Pythonu. Také jsou v něm napsané systémové databáze MySQL a Oracle
    Database. Kvůli jeho rychlosti jsou v něm napsané knihovny pro jiné programovací jazyky jako je
    NumPy, OpenGL či stb\_image.
    \cite{the_c_book:_featuring_the_ansi_c_standart, programming_in_c,
    top_applications_of_c_programming}\\ \\
    Jednoduchý program, který načte ze vstupu počet čísel, která chce uživatel setřídit. Následně
    daná čísla načte a vytiskne je seřazená:

\begin{lstlisting}[caption={sort\_n\_numbers.c},s label={lst:sort_n_numbers}, ref={lst:sort_n_numbers}]
#include <stdio.h>
#include <stdlib.h>

int compare(const void *a, const void *b)
{
    return (*(int *)a - *(int *)b);
}

int main(void)
{
    int sizeOfArray;
    scanf("%d", &sizeOfArray);
    int *array = malloc(sizeOfArray *sizeof(int));

    for(int i = 0; i < sizeOfArray; ++i)
        scanf("%d", &array[i]);

    qsort(array, sizeOfArray, sizeof(int), compare);

    for(int i = 0; i < sizeOfArray; ++i)
        printf("%d\n", array[i]);

    free(array);
    return 0;
}
\end{lstlisting}

\section{Programovací jazyk C++}
    work in progress \\
	Odkaz v závorkách: \parencite[see][page 900]{programming_in_c}\\
	Odkaz: \cite{programming_in_c}\\
	A odkaz pod čarou: \footcite[see][s. 42]{programming_in_c}\\
	Dobrý den, ahoj, \gls{atd}\\
	Praha, \gls{tj} hlavní město ČR
	
	\begin{table} [h!]
 		\caption{Testovací tabulka}
		\label{tab:test2}
			\begin{tabular}{ccccc}
				1 & 1 & 1  & 1  & 1  \\
				1 & 2 & 3  & 4  & 5  \\
				1 & 3 & 6  & 10 & 15 \\
				1 & 4 & 10 & 30 & 45
				\end{tabular}
	\end{table}


%%% v obsahu se objeví jen to co je v hranatých závorkách
    \begin{figure} [h!]
  \includegraphics[width=\linewidth]{test.jpg}
  \caption{Testovací}
  \label{fig:test}
\end{figure}
Obrázek \ref{fig:test} ukazuje Shangai z Pixabay.\\
Tabulka \ref{tab:test2} ukazuje hádejte, co.
	
\clearpage
\chapter{Grafická karta}			

\section{Grafická pipeline}

\section{Transformace a lineární algebra}

\section{Historie grafických karet}

%%%%%%%%%%%%%% PRAKTICKÁ ČÁST %%%%%%%%%%%%%%%%%%	
\part{Praktická část} % název praktické části (nenechávejte název Praktická část)

\section{Výpisy použitých programů}

Výpis programu \nameref{lst:hello_world}  naleznete ve výpise \ref{lst:hello_world}.

\begin{lstlisting}[title={Program hello.c}, caption={hello.c}, label={lst:hello_world}]
#include <stdio.h>
#define CISLO 10

int main(void) {
	int i = CISLO;

	print("Hello World!\n");
	print("%d", i);

	return (0);
}
\end{lstlisting}

\begin{lstlisting}[numbers=none, title={Příklad výstupního souboru}]
11.0524
5.5954
6.7996
13.8584
15.1357
Soucet: 52.4415
\end{lstlisting}

%%%%%%%%%%%%% ZÁVĚR
\chapter*{Závěr}
	
\nocite{*}
\printbibliography					% Vytvoří seznam literatury
\addcontentsline{toc}{chapter}{Bibliografie}
\printglossary[title={Zkratky}]		% Vytvoří seznam zkratek
\listoffigures						% Vytvoří seznam obrázků
\listoftables						% Vytvoří seznam tabulek

%%%%%%%%%%%%% PŘÍLOHY - APPENDIX 	
\begin{appendices}
	\chapter{Fotky z pokusů}	
	\lipsum[1]
    	%\pitem{Fotky z pokusů}
    	%\eitem{Vlastní program}
    	%\eitem{Dokumentace}
    	%\eitem{Testovací data}
	\chapter{Příloha další }
\end{appendices}
%%%%%%%%%%%%%%%
\end{document}
%%%%%%%%%%%%%%%%%%%% KONEC %%%%%%%%%%%%%%%%%%%%%%%%%
