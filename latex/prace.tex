%šablona pro maturitní práci Gymnázia Jírovcova 8, České Budějovice
% Autoři šablony: Jonáš Havelka, Michal Kočer, Daniel Sýkora
% Typ dokumentu: report
% veškeré úpravy v soubor MP.sty (styl maturitní práce)
\documentclass[12pt]{report}
% %%%%%%%%%%%%%%%%%%%%%%%%%%%%%%%%%%%%%%%%%%%%%%%%%%%%%%%
\usepackage{config}						  % Import stylu maturitní práce
\author{Arnošt Thor}                  % AUTOR PRÁCE
\title{Jak psát práci v \LaTeX{}u}    % NÁZEV PRÁCE
\date{14. února 2025}                 % DATUM ODEVZDÁNÍ PRÁCE
\vedouci{Mgr. František Kantor, PhD.} % VEDOUCÍ PRÁCE
\place{V Českých Budějovicích}
\skolnirok{2025/2026}                  % ŠKOLNí ROK
\logo{\includegraphics[scale=1.25]{GJ8_logotyp}} %Logo školy
%%%%%%%%%%%%%%%%%%%%%%%%%%%%%%%%%%%%%%%%%%%%%%%%%%%%%%%%%%%%%%%%%%%
\begin{document} %%%%%%% začátek dokumentu
%%%%%%%%%%%%%%%%%%%%%%%%%%%%  Titulní stránka + úvodní povinné stránky
\pagenumbering{roman}                   % číslování stránek římskými číslicemi
	\mytitlepage						% Vygenerování titulní strany
	
	\prohlaseni{
		Prohlašuji, že jsem tuto práci vypracoval samostatně s vyznačením všech použitých pramenů.
	}	
	
	\abstrakt{
	}{
	}
	
	\podekovani{
	}
	
   {\tableofcontents\newpage}			% Obsah
	
\addtocounter{page}{1}		% Posunutí countru stránek
\pagenumbering{arabic}		% Číslování stránek arabskými číslicemi
\chapter*{Úvod}     % úvod práce 
\addcontentsline{toc}{chapter}{Úvod}

%%%%%%%%%%%%%% TEORETICKÁ ČÁST %%%%%%%%%%%%%%%%%%	
\part{Teorie k vývoji hry v OpenGL}  % název teoretické části (nenechávejte Teoretická část)
	
\chapter{Programovací jazyky}
			
\section{Programovací jazyk C}

    C je středněúrovňový programovací jazyk, tedy jazyk, který je podobou blízko strojovému kódu,
    ale má už prvky vyššího programovacího jazyka jako jsou funkce, datové struktury nebo to že je
    strukturovaný. Je kompilovaný a statický, což znamená, že se program musí nejdříve přeložit do
    strojového kódu a až pak se může spustit. Datové typy jsou známy v čase kompilace, proto všechny
    proměnné musí být v kódu deklarovány, jelikož vkládání vstupních dat do programu probíhá až při
    běhu programu. Programuje se v něm strukturovaně a procedurálně, tedy kód se píše pomocí
    řídících struktur(if, while, for atd.) a pomocí funkcí, které umožňují používat části
    kódu vícekrát. C nemá automatický správce paměti, takže je potřeba uvolňovat paměť manuálně. C
    má střídmou standardní knihovnu, která obsahuje základní matematické operace a funkce pro
    práci s pamětí a soubory, takže jakékoliv složitější datové struktury či funkce si člověk musí
    naprogramovat sám. Tato strohost a blízkost ke strojovému kódu z C dělá jeden z nejrychlejších
    programovacích jazyků.
    \cite{programming_in_c, the_c_programming_language, features_of_C,
    low-level_midlvel_and_high-level_language, structured_programming}

    C bylo vytvořeno Dennisem Ritchiem na počátku 70. let 20. století v AT\&T Bellových 
    laboratořích. Jeho předchůdci byly jazyky ALGOL, CPL, BCPL a B. Jeho prvotním účelem bylo
    přepsat operační systém UNIX do použitelnějšího jazyka než Assembly a B. Už koncem 70. let bylo
    C populární, ale nebylo standardizované a vznikalo mnoho různých variant. Na začátku 80. let
    tedy Americký národní institut pro standardy(ANSI) zahájil práci na formální standardizované
    verzi. Tu dokončili v roce 1989 a je známa pod jménem C89. V průběhu let vycházely další verze,
    které jazyk zlepšovaly a modernizovali. Nejdůležitější verze byly C99, C11 a C17. Norma C23 byla
    nedávno schválena a teď se implementuje do kompilátorů. V součastnosti mezi nejpoužívanější
    kompilátory patří GCC a Clang. Jelikož bylo C velice populární,
    ovlivnilo řadu jiných programovacích jazyků, jako C\+\+, C\#, Java, Rust, Go atd.
    \cite{the_c_book:_featuring_the_ansi_c_standart, c_a_reference_manual,
    programming_in_c, the_c_programming_language}

    POZNAMKA pridej jeste k historii historii compileru, at tam nejsou jen soucanse.

    C je univerzální programovací jazyk, má tedy širokou škálu využití. Jeho první využití bylo k
    napsání UNIXu, který později ovlivnil operační systémy jako Linux, macOS, iOS a Android.
    Používá se v programování softwaru s omezenou pamětí a výkonem, jako je firmware aut či v
    zařízeních chytrých domácností. Dále se využívá pro tvoření kompilátorů a interpreterů jako je
    GCC nebo interpreter Pythonu. Také jsou v něm napsané systémové databáze MySQL a Oracle
    Database. Kvůli jeho rychlosti jsou v něm napsané knihovny pro jiné programovací jazyky jako je
    NumPy, OpenGL či GLFW.
    \cite{the_c_book:_featuring_the_ansi_c_standart, programming_in_c,
    top_applications_of_c_programming}

    Jednoduchý program, který načte ze vstupu počet čísel, která chce uživatel setřídit. Následně
    daná čísla načte a vytiskne je seřazená:

\begin{lstlisting}[caption={sort\_n\_numbers.c}]
#include <stdio.h>
#include <stdlib.h>

int compare(const void *a, const void *b)
{
    return (*(int *)a - *(int *)b);
}

int main(void)
{
    int sizeOfArray;
    scanf("%d", &sizeOfArray);
    int *array = malloc(sizeOfArray *sizeof(int));

    for(int i = 0; i < sizeOfArray; ++i)
        scanf("%d", &array[i]);

    qsort(array, sizeOfArray, sizeof(int), compare);

    for(int i = 0; i < sizeOfArray; ++i)
        printf("%d\n", array[i]);

    free(array);
    return 0;
}
\end{lstlisting}

\section{Programovací jazyk C++}
    Programovací jazyk C++ je v mnoha ohledech podobný jazyku C. Je stejně jako C středně\-úrovňový,
    kompilovaný, statický, má datové typy známy v době kompilace a nemá automatický správce
    paměti. V C++ se také programuje strukturovaně a procedurálně, ale na rozdíl od C také umožňuje
    programovat objektově. Objektové programování umožňuje používat objekty, které jsou vytvořeny
    pomocí tříd. Tyto třídy a jejich hierarchie děljí kód přehlednější a usnadňuje budoucí
    rozšiřování a debuggování. Dalším rozdílem je standardní knihovna, kterou má C++ rozsáhlejší.
    Obsahuje nové kontejnery jako vector, map, a priority\_queue, které jsou tvořeny pokročilejšími
    datovými strukturami jako binární vyhledávací strom nebo heap. Dále obsahuje nové algoritmy,
    například sort, find nebo count. Kvůli velké podobnosti C a C++ se často může C kód používat v
    C++, ale není tomu tak vždy. Například tento kód:
\begin{lstlisting}[caption={incompatibility\_example.c}]
int class(int new, int bool); 
\end{lstlisting}
    V C tento kód vytvoří funkci class, která vrací int a má 2 parametry new a bool. V C++ jsou ale
    class, new a bool klíčová slova, která nelze použít v názvu proměnných a funkcí. Proto se
    doporučuje programovat v C tak, aby daný C kód byl podmnožinou C++, což umožňuje jeho použití v
    jiném C++ programu. \cite{the_cpp_programming_language, programming_principles_and_practice_using_cpp}

    C++ bylo vytvořeno Bjarnem Stroustrupem v roce 1979 v AT\&T Bellových laboratořích. Před 
    vytvořením C++ pracoval Stroustrup s programovacím jazykem Simula 67, který byl
    oběktově-orientovaný a sloužil primárně k vytváření simulací. Stroustupovi přišlo
    objektově-orientované velmi užitečné, ale Simula 67 byl příliš pomalý pro větší projetky. Proto
    se rozhodl že vytvoří nadmnožina jazyka C, která by umožňovala objektově-orientované 
    programování a zároveň si zachovalal rychlost C, s názvem C with Classes.
\clearpage
\chapter{Grafická karta}			

\section{Historie grafických karet}

\section{Architektura grafických karet}

\section{Grafická pipeline}

\section{Transformace a lineární algebra}

\chapter{OpengGL}

\section{Historie OpenGL}

\section{OpenGL pipeline}

\section{Shadery}

\section{Textury}

%%%%%%%%%%%%%% PRAKTICKÁ ČÁST %%%%%%%%%%%%%%%%%%	
\clearpage
\part{Vývoj hry v OpenGL} % název praktické části (nenechávejte název Praktická část)

\chapter{Grafika a zvuk}

\section{Aseprite}

\section{Bosca Ceoil}

\chapter{Použité knihovny}

\section{GLAD}

\section{GLFW}

\section{stb\_image}

\chapter{Herní scény}

\section{Scéna hlavního menu}

\section{Scéna pozastavené hra}

\section{Scéna hry}

\chapter{Serní mechaniky}

\section{Generování objektů}

\section{Pohyb hráče}

\section{Detekce kolize}

%%%%%%%%%%%%% ZÁVĚR
\chapter*{Závěr}
	
\nocite{*}
\printbibliography					% Vytvoří seznam literatury
\addcontentsline{toc}{chapter}{Bibliografie}
\printglossary[title={Zkratky}]		% Vytvoří seznam zkratek
\listoffigures						% Vytvoří seznam obrázků
\listoftables						% Vytvoří seznam tabulek

%%%%%%%%%%%%% PŘÍLOHY - APPENDIX 	
\begin{appendices}
	\chapter{Fotky z pokusů}	
	\lipsum[1]
    	%\pitem{Fotky z pokusů}
    	%\eitem{Vlastní program}
    	%\eitem{Dokumentace}
    	%\eitem{Testovací data}
	\chapter{Příloha další }
\end{appendices}
%%%%%%%%%%%%%%%
\end{document}
%%%%%%%%%%%%%%%%%%%% KONEC %%%%%%%%%%%%%%%%%%%%%%%%%
